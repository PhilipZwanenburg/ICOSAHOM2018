%%%%%%%%%%%%%%%%%%%%%%%%%%%%%%%%%%%%%%%%%%%%%%%%%%%%%%%%%%%%%
\subsubsection{DPG}

%%%%%%%%%%%%%%%%%%%%%%%%%
\begin{frame}
\frametitle{The Discontinuous Petrov-Galerkin (DPG) Method}

Demkowicz et al. (2010)~\cite{Demkowicz2011}.
\\~

Motivated by optimal solution convergence in the $\fspace{L}^2$ norm, we would like
to move gradients in the bilinear form to the test space.
\begin{align}
b(v,u) = b^*(v,u) + c(\text{tr}_A^* v, \text{tr}_A u).
\end{align}

Graph space for the adjoint:
\begin{align}
\fspace{H}_b^*(\Omega) \coloneqq
\{ v \in (L^2(\Omega)) : B^*v \in (L^2(\Omega))\ \forall u \in \fspace{U} \}.
\end{align}

When setting $\fspace{V} = \fspace{H}_b^*(\Omega)$, we say that the test space
is $\fspace{H}_b$-conforming.

\end{frame}

%%%%%%%%%%%%%%%%%%%%%%%%%
\begin{frame}
\frametitle{The DPG Method - Broken Test Space}

No assumptions yet made regarding conformity of trial and test spaces.
\\~

% The goal of the methodology is to solve for the solution over a tessellation,
% $\mathcal{T}_h$, of the discretized domain, $\Omega_h$, consisting of elements
% (referred to as volumes), $\fe{V}$.
% \\~

To make the method practical, the DPG method uses broken test spaces such that
test functions can be computed elementwise,
\begin{align} \label{eq:auxiliary_opt_v_infinite}
\text{Find}\ v_{T u_h} \in \fspace{V}(\fe{V})\ \text{such that}\
(w,v_{T u_h})_{\fspace{V}(\fe{V})} = b(w,u_h),\ \forall w \in \fspace{V}(\fe{V})
\end{align}

where
\begin{align}
\fspace{V}(\Omega_h)
&\coloneqq
\{ v \in L^2(\Omega) : v|_{\fe{V}} \in \fspace{H}_b^*(\fe{V})\
   \forall \fe{V} \in \mathcal{T}_h \}, \\
(w,v)_{\fspace{V}(\Omega_h)}
&\coloneqq
\sum_{\fe{V}} (w|_{\fe{V}},v|_{\fe{V}})_{\fspace{V}(\fe{V})}.
\end{align}

\end{frame}

%%%%%%%%%%%%%%%%%%%%%%%%%
\begin{frame}
\frametitle{The DPG Method - Additional}

Investigated norms:
\\~

$\fspace{H}^1_\varv{b}: (w|_{\fe{V}},v|_{\fe{V}})_{\fspace{V}(\fe{V})}
  = (w|_{\fe{V}},v|_{\fe{V}}) +(\varv{b} \cdot \nabla w|_{\fe{V}},\varv{b} \cdot \nabla v|_{\fe{V}})$.
\\~

$\fspace{H}^-_\varv{b}: (w|_{\fe{V}},v|_{\fe{V}})_{\fspace{V}(\fe{V})} =
  h \langle w|_{\fe{V}},|\varv{b} \cdot \varv{\hat{n}}| v_{\fe{V}} \rangle_{\partial \fe{V}^-} +(\varv{b} \cdot \nabla w|_{\fe{V}},\varv{b} \cdot \nabla v|_{\fe{V}})$.
\\~

Additional characteristics:
\begin{itemize}
  \item General Petrov-Galerkin methodology outlined in the abstract setting is
    a subset of the practical DPG methodology. When both formulations are
    uniquely solvable, their solutions coincide.
  \item The required selection of a suitable numerical flux has been replaced
    with the required selection of a suitable test norm.
  \item When also using a discontinuous trial space, additional trace unknowns
    are introduced allowing for static condensation of volume unknowns in the
    global solve.
\end{itemize}

\end{frame}
