% ICOSAHOM London July 2018 Conference : Presentation
% Philip Zwanenburg, McGill University, Montreal, QC, Canada

\documentclass[]{article}



\PassOptionsToPackage{usenames,dvipsnames}{xcolor}
\usepackage{lmodern}
\usepackage{amsmath,amsthm,mathtools,bm}


\usepackage{hyperref}
\hypersetup{ colorlinks, citecolor=black, filecolor=black, linkcolor=blue, urlcolor=blue, }

\usepackage{subcaption}

\usepackage{amsmath}
\numberwithin{equation}{section}
\mathtoolsset{showonlyrefs} % Only show numbers if equations are referred to.


\setlength\abovecaptionskip{1pt}
\setlength\belowcaptionskip{1pt}

% For Theorems, Lemmas, Proofs, etc; 
% http://www.maths.tcd.ie/~dwilkins/LaTeXPrimer/Theorems.html
\usepackage{amsthm}
%\newtheorem{theorem}{Theorem}[section]
%\newtheorem{lemma}[theorem]{Lemma}
\newtheorem{proposition}[theorem]{Proposition}
\newtheorem*{remark}{Remark}
%\newtheorem{corollary}[theorem]{Corollary}

% Notation (math)
\newcommand{\vars}[1]{\mathit{{{#1}}}}    % 's'calar 'var'iable
\newcommand{\varv}[1]{\mathit{\bm{{#1}}}} % 'v'ector 'var'iable
%\newcommand{\varg}[1]{\mathit{\bm{{#1}}}} % 'g'roup  'var'iable

\newcommand{\vect}[1]{\bm{{#1}}}
\newcommand{\mat}[1]{\bm{{#1}}}
\newcommand{\tens}[1]{\bm{{#1}}}
%\newcommand{\vect}[1]{\underline{{#1}}}
%\newcommand{\mat}[1]{\underline{\overline{{#1}}}}
%\newcommand{\tens}[1]{\underline{\underline{\bm{#1}}}}

\let\oldnabla\nabla
\renewcommand{\nabla}{{\vect{\oldnabla}}}


\newcommand{\fspace}[1]{\mathit{{{#1}}}}                     % 'f'unction space
\newcommand{\ftspace}[1]{\hat{\mathit{{{#1}}}}}              % 'f'unction 't'race space
\newcommand{\afspace}[1]{\tilde{\mathit{{{#1}}}}}            % 'a'pproximate 'f'unction space
\newcommand{\aftspace}[1]{\widetilde{\hat{\mathit{{{#1}}}}}} % 'a'pproximate 'f'unction 't'race space
\newcommand{\fe}[1]{\mathcal{{{#1}}}}                        % 'f'inite 'e'lement

% Highlighting
\usepackage[usenames,dvipsnames]{xcolor}
\newcommand{\comment}[1]{{\color{red}#1}}
\newcommand{\makered}[1]{{\color{red}#1}}
\newcommand{\makegreen}[1]{{\color{ForestGreen}#1}}
\newcommand{\makeblue}[1]{{\color{blue}#1}}


\usepackage[margin=0.5in]{geometry}

\begin{document}

\title{On the Selection of Finite Element Test Norms based on their Optimal Stabilization and Convergence Properties}
\author{
Philip Zwanenburg \and
Siva Nadarajah
}

\maketitle

The success of standard Galerkin finite element methods for the solution of
elliptic partial differential equations (PDEs) arises from the relation between
the variational formulations and the minimization of an energy functional.
In the case of convection-dominated or purely hyperbolic PDEs, additional
stabilization is required in regions of high local P\'eclet or Reynolds number.
While the need for this stabilization is understood, the specific form which it
should take in order to guarantee both stable and optimally convergent
discretizations is not generally provided.
The goal of this work is to devise schemes for which the stabilization is
optimal in the sense that the computed solution is given by the $\fspace{L}^2$ projection of
the exact solution.

Consider the linear abstract variational problem
\begin{align} \label{eq:bilinear_abstract_infinite}
\text{Find}\ u \in \fspace{U}\ \text{such that}\
b(v,u) = l(v)\ \forall v \in \fspace{V}.
\end{align}

We have the standard requirements:
\begin{align}
| b(v,u) | \le M
||v||_{\fspace{V}}
||u||_{\fspace{U}},
\end{align}
\begin{align} \label{eq:inf_sup_infinite}
\exists \gamma > 0 :
\adjustlimits\inf_{u\in\fspace{U}} \sup_{v\in\fspace{V}}
\frac{b(v,u)}{||v||_{\fspace{V}}||u||_{\fspace{U}}} \ge \gamma,
\end{align}

and
\begin{align}
l(v) = 0\ \forall v \in \fspace{V}_0,\ \text{where}\ \fspace{V}_0 \coloneqq \{v \in \fspace{V} : b(v,u) = 0\ \forall u
\in \fspace{U} \}.
\end{align}

Then by the Banach-Ne\v{c}as-Babu\v{s}ka theorem,
\begin{align}
||u||_{\fspace{U}} \le \frac{M}{\gamma} ||l||_{\fspace{V}'}.
\end{align}

Choosing finite dimensional subsets we have the discrete variational problem,
\begin{align} \label{eq:bilinear_abstract_finite}
\text{Find}\ u_h \in \fspace{U}_h\ \text{such that}\
b(v_h,u_h) = l(v_h)\ \forall v_h \in \fspace{V}_h.
\end{align}

If a discrete version of the inf-sup condition holds,
\begin{align} \label{eq:abstract_dpg_error}
||u-u_h||_{\fspace{U}} \le \frac{M}{\gamma_h} \inf_{w_h \in \fspace{U}_h} ||u-w_h||_{\fspace{U}}.
\end{align}

Several points are now of note. First, it can be seen that the computed solution
has an error proportional to the infimum of that of all possible discrete
solutions \emph{in the trial space norm} set according to the continuity and
inf-sup conditions. If the trial norm is given by the $\fspace{L}^2$ norm and
the continuity and inf-sup constants are independent of the mesh size and
solution regime, it is thus observed that the computed solution will converge
optimally in the L2 norm. Second, a case of particular interest occurs when the
continuity and inf-sup constants are equal
\begin{align} \label{eq:equal_m_gamma}
M = \gamma_h,
\end{align}

where the error incurred by the discrete approximation is smallest. It has been
proven that

\begin{align} \label{eq:bui_thanh_2-6}
  b(v,u) \le ||v||_\fspace{V} ||u||_\fspace{U}\ \text{and}\
 \exists v_u \in \fspace{V}\ \setminus \{0\} :
b(v_u,u) = ||v_u||_{\fspace{V}} ||u||_{\fspace{U}}\ \forall u \in \fspace{U}
\implies \gamma_h = 1\ (= M)
\end{align}

where $v_u$ is termed an optimal test function for the trial function $u$.
Further, the recentrly formulated optimal trial Petrov-Galerkin (OPG)
method has been devised to satisfy the requirements of
this theorem~\cite{Brunken2018}.
\iffalse
Once a test norm is selected, we can define the map from trial to test space,
$T: \fspace{U} \ni u \rightarrow Tu \coloneqq v_{Tu} \in \fspace{V} $,
using the test space inner product, by
\begin{align}
(v, v_{Tu})_{\fspace{V}} = b(v,u),
\end{align}

such that the optimal test space, achieving
the supremum in the discrete inf-sup constant, is given by the solution to the
following linear system for each of the trial functions
\begin{align} \label{eq:def_discrete_test_space_alt}
\fspace{V}_\text{opt} = \{ v_{T u} \in \fspace{V} : u \in \fspace{U} \}.
\end{align}

In the discrete trial space, optimal test functions are determined according to
\begin{align} \label{eq:auxiliary_opt_v}
\text{Find}\ v_{T u_h} \in \fspace{V}\ \text{such that}\
(w,v_{T u_h})_{\fspace{V}} = b(w,u_h),\ \forall w \in \fspace{V}.
\end{align}

Note that the continuous test space is still used above and that the specific
choice of discrete test space must be made according to the \emph{practical}
realization of the method.
\fi
With the above abstract framework in place, we now move
on to a specific model problem providing a concrete realization of the concepts
presented above. Consider the linear advection equation,

\begin{alignat}{3}
& \varv{b} \cdot \nabla \vars{u} = s\ && \text{in}\ \Omega, \label{eq:linear_advection_omega} \\
& u = u_{\Gamma^{\text{-}}}\ && \text{on}\ \Gamma^{\text{-}} \coloneqq \{ \varv{x} \in \partial\Omega : \varv{\hat{n}} \cdot \vect{b} < 0 \}.
\end{alignat}

In the standard manner, we obtain the weak formulation where the bilinear and
linear forms are defined by
\begin{align} \label{eq:bilinear_pg_advection_jump}
b(v,\varg{u})
=\ \sum_{\fe{V}} \int_{\fe{V}} - \nabla v \cdot \varv{b} u\ d\omega
+ \sum_{\fe{F}} \int_{\fe{F}} \jump{v} f^*\ d\gamma,\
l(v)
=\ \sum_{\fe{V}} \int_{\fe{V}} v s\ d\omega.
\end{align}

The OPG method corresponds to making the choice of trial space norm as the
$\fspace{L}^2$ norm and achieves unity continuity and inf-sup conditions.
Applying Cauchy-Schwarz to the bilinear form
\begin{align} \label{eq:cauchy_scwarz_advection}
b(v,\varg{u})
\le\ & \sum_{\fe{V}}
|| - \nabla v \cdot \varv{b} ||_{L^2(\fe{V})} || u ||_{L^2(\fe{V})}
+
||\jump{v}||_{L^2(\fe{F})} ||f^*||_{L^2(\fe{F})} \\
\le\ &
\underbrace{\left(
\sum_{\fe{V}} || - \nabla v \cdot \varv{b} ||^2_{L^2(\fe{V})} + ||\jump{v}||^2_{L^2(\fe{F})}
\right)^\frac{1}{2}}_{||v||_\fspace{V}}
       \times
\underbrace{\left(
\sum_{\fe{V}} || u ||^2_{L^2(\fe{V})} + ||f^*||^2_{L^2(\fe{F})}
\right)^\frac{1}{2}}_{||u||_\fspace{U}},
\end{align}

we naturally find that equality is obtained for the choice of test functions
\begin{alignat}{3}
u =\ & -\nabla v_u \cdot \vect{b}\quad && \text{in}\ \fe{V}, \\
f^* =\ & \jump{v_u} && \text{on}\ \fe{F}.
\end{alignat}

Imposing sufficient conditions on the test space, notably regarding the
selection of the discrete basis and the imposition of adjoint boundary
conditions (on the outflow faces of the domain), we have
\begin{align}
  \exists\ \text{a unique}\ v_u \in \fspace{V}\ \text{such that}\ u = B^*v_u,\ \forall u \in \fspace{U}.
\end{align}

An equivalent problem to the original is then be obtained
where now it is the test functions \emph{having global
support} which are solved for and from which the solution is then computed
through the application of the adjoint operator,
\begin{align}
& \text{Find}\ w_h(u_h) \in \fspace{V}_h\ \text{such that}\\
& b(v_h,u_h) \coloneqq b(v_h,B^*w_h) \coloneqq (B^*v_h,B^*w_h) = l(v_h)\ \forall v_h \in \fspace{V}_h.
\end{align}

That the problem remains practical despite the global support of the test
functions is in stark contrast to the necessary limited support for the test
functions in the discontinuous Petrov-Galerkin (DPG) method~\cite{Demkowicz2010}
which served as inspiration for this formulation. While the OPG method as
presented above has the appearance of achieving the optimal stability which is
the focus of this study, we note several open problems which we have encountered
during our initial implementation:
\begin{itemize}
  \item The degree of the test space is required to be one higher than that of
    the trial space, leading to a global system matrix corresponding to a
    problem of one higher degree;
  \item Simply imposing the physically motivated zero Dirichlet boundary
    conditions along all outflow boundaries results in the values of the solution
    being forced to be incorrect at certain points on the boundary
    when a polynomial discrete test space is used. Thus even for the simple
    problem of advection of a constant solution along a diagonal advection
    field, all of the standard polynomial spaces typically used in finite
    element contexts are overly restrictive.
\end{itemize}

We note that the additional stability provided by the method may provide a
motivation for accepting the increased cost resulting from the higher degree of
the test space representation. Further, we note that it may be possible to
remedy the forcing of incorrect solution values resulting from the test space
boundary conditions by extending the solution in a smooth manner outside of the
domain, but that this will likely lead to complications in more complex
settings. Finally, we expect to extend the OPG method to nonlinear equations
(beginning with Burgers' equation) by solving a series of linearized problems
around the most recently computed state.

\bibliographystyle{ieeetr}
\bibliography{../backmatter/global}

\end{document}