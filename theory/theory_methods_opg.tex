%%%%%%%%%%%%%%%%%%%%%%%%%%%%%%%%%%%%%%%%%%%%%%%%%%%%%%%%%%%%%
\subsubsection{OPG}

%%%%%%%%%%%%%%%%%%%%%%%%%
\begin{frame}
\frametitle{The Optimal Trial Petrov-Galerkin (OPG) Method}

Brunken et al. (2018)~\cite{Demkowicz2011}.
\\~

Similar motivation to that of the DPG method but with test norm
chosen as that which is naturally induced by the problem such that:
\begin{itemize}
\item the trial norm corresponds to the L2 norm, and
\item the continuity and inf-sup constants take values of 1!
\end{itemize}

\end{frame}

%%%%%%%%%%%%%%%%%%%%%%%%%
\begin{frame}
\frametitle{The OPG Method - Formulation}

Cauchy-Schwarz on bilinear form:
\begin{align} \label{eq:cauchy_scwarz_advection}
b(v,\varg{u})
\le\ & \sum_{\fe{V}}
|| - \nabla v \cdot \varv{b} ||_{L^2(\fe{V})} || u ||_{L^2(\fe{V})}
+
||\jump{v}||_{L^2(\fe{F})} ||f^*||_{L^2(\fe{F})} \\
\le\ &
\underbrace{\left(
\sum_{\fe{V}} || - \nabla v \cdot \varv{b} ||^2_{L^2(\fe{V})} + ||\jump{v}||^2_{L^2(\fe{F})}
\right)^\frac{1}{2}}_{||v||_\fspace{V}}
       \times \\
  &
\underbrace{\left(
\sum_{\fe{V}} || u ||^2_{L^2(\fe{V})} + ||f^*||^2_{L^2(\fe{F})}
\right)^\frac{1}{2}}_{||u||_\fspace{U}}.
\end{align}

Equality obtained (unity continuity and inf-sup constants) for the choice of
test functions:
\begin{alignat}{3}
u =\ & -\nabla v_u \cdot \vect{b}\quad && \text{in}\ \fe{V}, \\
f^* =\ & \jump{v_u} && \text{on}\ \fe{F}.
\end{alignat}

\end{frame}

%%%%%%%%%%%%%%%%%%%%%%%%%
\begin{frame}
\frametitle{The OPG Method - Formulation}

Imposing sufficient constraints on the test space,
\begin{align}
  \exists\ \text{a unique}\ v_u \in \fspace{V}\ \text{such that}\ u = B^*v_u,\ \forall u \in \fspace{U}.
\end{align}

The equivalent problem can then be obtained:
\begin{align}
& \text{Find}\ w_h(u_h) \in \fspace{V}_h\ \text{such that}\\
& b(v_h,u_h) \coloneqq b(v_h,B^*w_h) \coloneqq (B^*v_h,B^*w_h) = l(v_h)\ \forall v_h \in \fspace{V}_h,
\end{align}

and subsequently compute the solution.
\\~

Main difference with DPG: solve for the optimal test function, having global
support, followed by the computation of discrete trial space through the
application of the adjoint operator.

\end{frame}
