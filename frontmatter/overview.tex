%%%%%%%%%%%%%%%%%%%%%%%%%%%%%%%%%%%%%%%%%%%%%%%%%%%%%%%%%%%%%
\section{Overview}

%%%%%%%%%%%%%%%%%%%%%%%%%
\begin{frame}
\frametitle{Motivation and Goals}

The success of Bubnov-Galerkin finite element methods for the solution of
elliptic partial differential equations (PDEs) arises from the relation between
the variational formulations and the minimization of an energy functional.
\\~

In the case of convection-dominated or purely hyperbolic PDEs, additional
stabilization is required in regions of high local P\'eclet number and is most
commonly introduced either through a suitably chosen numerical flux or through
the modification of the test space (resulting in a Petrov-Galerkin method).
\\~

Taking the linear advection equation as a model problem, our goal is to present
our initial investigation into determining the \emph{optimal} form of
stabilization.

\end{frame}

%%%%%%%%%%%%%%%%%%%%%%%%%
\begin{frame}
\frametitle{Outline}

\begin{itemize}
\item Presentation of abstract functional setting and methods considered:
\begin{itemize}
\item Discontinuous Petrov-Galerkin method with various test norms~\cite{Demkowicz2011};
\item Optimal Trial Petrov-Galerkin method~\cite{Brunken2018};
\item Discontinuous Galerkin method with upwind numerical flux.
\end{itemize}
\item Numerical results in one and two dimensions.
\item Discussion of limitations and future directions.
\end{itemize}

\end{frame}