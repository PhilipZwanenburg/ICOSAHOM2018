%%%%%%%%%%%%%%%%%%%%%%%%%%%%%%%%%%%%%%%%%%%%%%%%%%%%%%%%%%%%%
\section{Conclusion}

%%%%%%%%%%%%%%%%%%%%%%%%%
\begin{frame}
\frametitle{Concluding Remarks}

\begin{itemize}
  \item The relation between trial/test norms and solution convergence in
    $\fspace{L}^2$ offers very interesting possibilities for the formulation of
    stable and optimally convergent numerical methods.
  \item Despite the advantageous properties, the Petrov-Galerkin methods considered
    here had several serious drawbacks in relation to their required discrete spaces.
  \item It will be important to find situations in which the improved stability
   justifies the added cost if these methods are ever to be of value.
\end{itemize}

\end{frame}

%%%%%%%%%%%%%%%%%%%%%%%%%
\begin{frame}
\frametitle{Future Directions}

\begin{itemize}
  \item Extension of ideas to nonlinear problems:
\begin{itemize}
  \item Attempting to formulate norms which
    have better properties than those previously proposed;
  \item Attempting to draw motivation from numerical fluxes used for DG-type
    methods through a similar interpretation of stabilization in terms of
    penalization of face terms in the norm used to establish coercivity.
\end{itemize}
  \item Investigation of a relation to nonlinear stability (energy/entropy stability).
\end{itemize}

\vspace{1cm}

We gratefully acknowledge the financial support of NSERC and McGill University.

\end{frame}