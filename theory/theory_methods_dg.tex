%%%%%%%%%%%%%%%%%%%%%%%%%%%%%%%%%%%%%%%%%%%%%%%%%%%%%%%%%%%%%
\subsubsection{DG}

%%%%%%%%%%%%%%%%%%%%%%%%%
\begin{frame}
\frametitle{The Discontinuous Galerkin (DG) Method}

The DG method with upwind numerical flux corresponds to:
\begin{align}
b(v,u)
=\ & \sum_{\fe{V}} \int_{\fe{V}} - \nabla v \cdot \varv{b} u\ d\omega
+ \sum_{\fe{F}} \int_{\fe{F}} \jump{v} f^*\ d\gamma,
\end{align}

where $f^* = \varv{\hat{n}} \cdot \varv{b} u_\text{upwind}$. Equivalently,
\begin{align}
b(v,u)
=\ & \sum_{\fe{V}} \int_{\fe{V}} - \nabla v \cdot \varv{b} u\ d\omega \\
  +\ & \sum_{\fe{F}} \int_{\fe{F}} \jump{v}
       \left(
       (\varv{b} \cdot \varv{\hat{n}}) \avg{u} +
       \frac{1}{2} |\varv{b} \cdot \varv{\hat{n}}| \jump{u}
       \right) d\gamma.
\end{align}

\begin{itemize}
  \item Central flux term for discrete coercivity in $||v||^2_\text{cf} =
    ||v||^2_{\fspace{L}^2(\Omega)} + \int_{\partial \Omega} \frac{1}{2}
    |\varv{b} \cdot \varv{\hat{n}}| v^2\ d\Gamma$.
  \item Additional penalization term added to strengthen the stability such that
    improved error estimates are obtained.
\end{itemize}

\end{frame}

%%%%%%%%%%%%%%%%%%%%%%%%%
\begin{frame}
\frametitle{The DG Method - Coercivity Norms}

Discrete coercivity leading to a quasi-optimal error estimate can be proven
using the following norm
\begin{align}
  ||v||^2_\text{uw$_1$}
  =\ &
  ||v||^2_{\fspace{L}^2(\Omega)}
  +
  \int_{\partial \Omega} \frac{1}{2} |\varv{b} \cdot \varv{\hat{n}}| v^2\ d\Gamma \\
  +\ &
  \sum_{\fe{F}} \int_{\fe{F}} \frac{1}{2} |\varv{b} \cdot \varv{\hat{n}}| \jump{v}^2 d\gamma
  +
  \sum_{\fe{V}} h_\fe{V} || \varv{b} \cdot \nabla v ||^2_{\fspace{L^2(\fe{V})}}.
\end{align}

Continuity requires the additional terms

\begin{align}
  ||v||^2_\text{uw$_2$}
  =
  ||v||^2_\text{uw$_1$}
  +
  \sum_{\fe{V}} \left( h_\fe{V}^{-1} || v ||^2_{\fspace{L^2(\fe{V})}} + || v ||^2_{\fspace{L^2(\partial \fe{V})}} \right).
\end{align}

Reference: Di Pietro et al.~\cite{DiPietro2011}.

\end{frame}
