%%%%%%%%%%%%%%%%%%%%%%%%%%%%%%%%%%%%%%%%%%%%%%%%%%%%%%%%%%%%%
\section{Theory}

%%%%%%%%%%%%%%%%%%%%%%%%%%%%%%%%%%%%%%%%%%%%%%%%%%%%%%%%%%%%%
\subsection{Abstract Setting}

%%%%%%%%%%%%%%%%%%%%%%%%%
\begin{frame}
\frametitle{Abstract Functional Setting - Continuous}

Consider the \emph{linear} abstract variational problem
\begin{align} \label{eq:bilinear_abstract_infinite}
\text{Find}\ u \in \fspace{U}\ \text{such that}\
b(v,u) = l(v)\ \forall v \in \fspace{V}.
\end{align}

We have the standard requirements:
\begin{align}
| b(v,u) | \le M
||v||_{\fspace{V}}
||u||_{\fspace{U}},
\end{align}
\begin{align} \label{eq:inf_sup_infinite}
\exists \gamma > 0 :
\adjustlimits\inf_{u\in\fspace{U}} \sup_{v\in\fspace{V}}
\frac{b(v,u)}{||v||_{\fspace{V}}||u||_{\fspace{U}}} \ge \gamma,
\end{align}

and
\begin{align}
l(v) = 0\ \forall v \in \fspace{V}_0,\ \text{where}\ \fspace{V}_0 \coloneqq \{v \in \fspace{V} : b(v,u) = 0\ \forall u
\in \fspace{U} \}.
\end{align}

Then by the Banach-Ne\v{c}as-Babu\v{s}ka theorem,
\begin{align}
||u||_{\fspace{U}} \le \frac{M}{\gamma} ||l||_{\fspace{V}'}.
\end{align}

\makered{Add elaboration of optimal convergence from Bathe based on inf-sup
  (more time )}

\end{frame}

%%%%%%%%%%%%%%%%%%%%%%%%%
\begin{frame}
\frametitle{Abstract Functional Setting - Discrete}

Choosing finite dimensional subsets we have the discrete variational problem,
\begin{align} \label{eq:bilinear_abstract_finite}
\text{Find}\ u_h \in \fspace{U}_h\ \text{such that}\
b(v_h,u_h) = l(v_h)\ \forall v_h \in \fspace{V}_h.
\end{align}

If a discrete version of the inf-sup condition holds,
\begin{align} \label{eq:abstract_dpg_error}
||u-u_h||_{\fspace{U}} \le \frac{M}{\gamma_h} \inf_{w_h \in \fspace{U}_h} ||u-w_h||_{\fspace{U}}.
\end{align}

\end{frame}

%%%%%%%%%%%%%%%%%%%%%%%%%
\begin{frame}
\frametitle{Abstract Functional Setting - Quick Proof}

Beginning with the discrete inf-sup condition we have, for all $w_h \in \fspace{U}_h$,
\begin{align}
  \gamma_h ||w_h-u_h||_\fspace{U}
  & \le \sup_{v_h\in\fspace{V_h}} \frac{b(v_h,w_h-u_h)}{||v_h||_\fspace{V}} \\
  & = \sup_{v_h\in\fspace{V_h}} \frac{b(v_h,w_h-u)+b(v_h,u-u_h)}{||v_h||_\fspace{V}} \\
  & = \sup_{v_h\in\fspace{V_h}} \frac{b(v_h,w_h-u)}{||v_h||_\fspace{V}} \\
  & \le \sup_{v_h\in\fspace{V_h}} \frac{M ||v_h||_\fspace{V}||w_h-u||_\fspace{U}}{||v_h||_\fspace{V}} \\
  & = M ||w_h-u||_\fspace{U}.
\end{align}

\end{frame}

%%%%%%%%%%%%%%%%%%%%%%%%%
\begin{frame}
\frametitle{Abstract Functional Setting - Quick Proof}

Using the triangle inequality and that above we thus find
\begin{align}
  ||u_h-u||_\fspace{U}
  & \le || u_h-w_h ||_\fspace{U} + || w_h-u ||_\fspace{U}\ && \forall w_h \in \fspace{U}_h \\
  & \le \frac{M}{\gamma_h} || w_h-u ||_\fspace{U} + || w_h-u ||_\fspace{U}\ && \forall w_h \in \fspace{U}_h \\
  & = \left(1 + \frac{M}{\gamma_h}\right) || w_h-u ||_\fspace{U}\ && \forall w_h \in \fspace{U}_h \\
  \rightarrow
  ||u_h-u||_\fspace{U} & \le \left(1 + \frac{M}{\gamma_h}\right) \inf_{w_h \in \fspace{U}_h}|| w_h-u ||_\fspace{U}.
\end{align}

\end{frame}

%%%%%%%%%%%%%%%%%%%%%%%%%
\begin{frame}
\frametitle{Abstract Functional Setting - Selection of Norms}

A case of particular interest:
\begin{align} \label{eq:equal_m_gamma}
M = \gamma_h.
\end{align}

\begin{itemize}
  \item Error incurred by the discrete approximation is smallest.
\end{itemize}
\vspace{5mm}

Bui-Thanh et al.~\cite[Theorem \makeblue{2.6}]{BuiThanh2013} have proven that
\begin{align} \label{eq:bui_thanh_2-6}
M = \gamma_h = 1
\iff
\exists v_u \in \fspace{V}\ \setminus \{0\} :
b(v_u,u) = ||v_u||_{\fspace{V}} ||u||_{\fspace{U}}\ \forall u \in \fspace{U},
\end{align}

where $v_u$ is termed an optimal test function for the trial function $u$.

\end{frame}

%%%%%%%%%%%%%%%%%%%%%%%%%
\begin{frame}
\frametitle{Abstract Functional Setting - Selection of Norms}

One choice of strategy is to select the norm which is naturally induced by the
problem. Defining the map from trial to test space,
$T: \fspace{U} \ni u \rightarrow Tu \coloneqq v_{Tu} \in \fspace{V} $, by
\begin{align}
(v_{Tu},Tu)_{\fspace{V}} = b(v,u),
\end{align}

then
\begin{align} \label{eq:def_discrete_test_space_alt}
\fspace{V}_\text{opt} = \{ v_{T u} \in \fspace{V} : u \in \fspace{U} \}.
\end{align}

In the discrete trial space optimal test functions are determined according to
\begin{align} \label{eq:auxiliary_opt_v}
\text{Find}\ v_{T u_h} \in \fspace{V}\ \text{such that}\
(w,v_{T u_h})_{\fspace{V}} = b(w,u_h),\ \forall w \in \fspace{V}.
\end{align}

\end{frame}

%%%%%%%%%%%%%%%%%%%%%%%%%%%%%%%%%%%%%%%%%%%%%%%%%%%%%%%%%%%%%
\subsection{Methods}

%%%%%%%%%%%%%%%%%%%%%%%%%%%%%%%%%%%%%%%%%%%%%%%%%%%%%%%%%%%%%
\subsubsection{DPG}

%%%%%%%%%%%%%%%%%%%%%%%%%
\begin{frame}
\frametitle{The Discontinuous Petrov-Galerkin (DPG) Method}

Motivated by optimal solution convergence in the $\fspace{L}^2$ norm, we would like
to move gradients in the bilinear form to the test space.
\\~

From the formal $L^2$-adjoint and a bilinear form representing the boundary terms
\begin{align}
b(v,u) = b^*(v,u) + c(\text{tr}_A^* v, \text{tr}_A u)
\end{align}

we obtrain the graph space for the adjoint
\begin{align}
\fspace{H}_b^*(\Omega) \coloneqq
\{ v \in (L^2(\Omega)) : b^*(v,u) \in (L^2(\Omega))\ \forall u \in \fspace{U} \}.
\end{align}

When setting $\fspace{V} = \fspace{H}_b^*(\Omega)$, we say that the test space is $\fspace{H}_b$-conforming.

\end{frame}

%%%%%%%%%%%%%%%%%%%%%%%%%
\begin{frame}
\frametitle{The Discontinuous Petrov-Galerkin Method}

No assumptions yet made regarding conformity of trial and test spaces.
\\~

The goal of the methodology is to solve for the solution over a tessellation,
$\mathcal{T}_h$, of the discretized domain, $\Omega_h$, consisting of elements (referred to as volumes), $\fe{V}$.
\\~

To make the method practical, the DPG method uses broken test spaces such that
test functions can be computed elementwise,
\begin{align} \label{eq:auxiliary_opt_v_infinite}
\text{Find}\ v_{T u_h} \in \fspace{V}(\fe{V})\ \text{such that}\
(w,v_{T u_h})_{\fspace{V}(\fe{V})} = b(w,u_h),\ \forall w \in \fspace{V}(\fe{V})
\end{align}

where
\begin{align}
\fspace{V}(\Omega_h)
&\coloneqq
\{ v \in L^2(\Omega) : v|_{\fe{V}} \in \fspace{H}_b^*(\fe{V})\ \forall \fe{V} \in \mathcal{T}_h \}, \\
(w,v)_{\fspace{V}(\Omega_h)}
&\coloneqq
\sum_{\fe{V}} (w|_{\fe{V}},v|_{\fe{V}})_{\fspace{V}(\fe{V})}.
\end{align}

\end{frame}

%%%%%%%%%%%%%%%%%%%%%%%%%
\begin{frame}
\frametitle{The Discontinuous Petrov-Galerkin Method}

Characteristics:
\begin{itemize}
  \item General Petrov-Galerkin methodology outlined in the abstract setting is
    a subset of the practical DPG methodology. When both formulations are
    uniquely solvable, their solutions coincide.
  \item The required selection of a suitable numerical flux has been replaced
    with the required selection of a suitable test norm.
  \item When also using a discontinuous trial space, additional trace unknowns
    are introduced allowing for static condensation of volume unknowns in the
    global solve.
\end{itemize}

\end{frame}

%%%%%%%%%%%%%%%%%%%%%%%%%%%%%%%%%%%%%%%%%%%%%%%%%%%%%%%%%%%%%
\subsubsection{OPG}

%%%%%%%%%%%%%%%%%%%%%%%%%
\begin{frame}
\frametitle{The Optimal Trial Petrov-Galerkin Method}

\makered{Follow the presentation of Di Pietro}.

\end{frame}

%%%%%%%%%%%%%%%%%%%%%%%%%%%%%%%%%%%%%%%%%%%%%%%%%%%%%%%%%%%%%
\subsubsection{DG}

%%%%%%%%%%%%%%%%%%%%%%%%%
\begin{frame}
\frametitle{Discontinuous Galerkin Method}

\makered{Follow the presentation of Di Pietro}.

\end{frame}

%%%%%%%%%%%%%%%%%%%%%%%%%%%%%%%%%%%%%%%%%%%%%%%%%%%%%%%%%%%%%
\subsection{Discrete inf-sup Testing}
